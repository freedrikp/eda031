\documentclass[a4paper]{article}

\usepackage[english]{babel}
\usepackage[utf8x]{inputenc}
\usepackage{amsmath}
\usepackage{graphicx}
\usepackage[colorinlistoftodos]{todonotes}

\title{Project Report \\ News System  \\ EDA031 C++ Programming}
\date{\today}
\author{Christian Andersen \\ dat11can@student.lu.se \and Ragnar Mellbin \\ dat11rmb@student.lu.se \and Fredrik Paulsson \\ dat11fp1@student.lu.se
\and Felix Åkerlund \\ dat11fak@student.lu.se}
%\setcounter{secnumdepth}{5}
%\setcounter{tocdepth}{5}
\begin{document}
\maketitle

%\begin{abstract}
%Your abstract.
%\end{abstract}

\section{Introduction}
In this project we are to develop a client/server news system. There will be two version of the server, one version that saves it's data in the primary memory and one version that saves the data on the disk memory instead. The client will be simple with only a text based interface.

The server will store a list of newsgrous each identified with a unique ID. For each newsgroup the server will store a list of articles. The articles will be identified with each own ID that is unique for the newsgroup to which the article belongs.

This report serves as documentation of our project.

\section{Description of system design}

\textbf{Enligt projektbeskrivningen: "A detailed description of your system design, both for the server and the clients. Preferably
use UML diagrams to give an overview of the design. (It is not necessary that you list
attributes and methods in these diagrams.) You must also describe the classes, at least as
far as stating the responsibilities of each class. Also give an overview of the dynamics of the server, i.e., trace an interaction between a
client and the server from the point that the server receives a command until it sends the
reply. UML sequence diagrams are good for this purpose."}

\subsection{Server}

Describes the common things between the servers

\subsubsection{In-Memory Server}

Describes the unique things in Memory server


\subsubsection{On-Disk Server}


Describes the unique things in Disk server


\subsection{Client}
The client is divided into four parts. These four parts consists of two classes, one struct and a file containing a few methods including the \texttt{main}-method of the client itself. Out of these four parts we have only developed two. An overview of the client parts is shown in figure \ref{clientUML}.

\begin{figure}
    \centering
    \includegraphics[width=0.6\textwidth]{projectUML-client.png}
    \caption{UML diagram client the client}
    \label{clientUML}
\end{figure}

We have developed the two parts named \texttt{Client} and \texttt{ClientMessageHandler}. The parts named \texttt{Connection} and \texttt{Protocol} were given to us when we started the development.

The \texttt{Connection} class simply describes a connection to the server part and has methods which allows us to write messages to the server. The \texttt{Protocol} struct defines constants used in the protocol that were specified for us.

The \texttt{ClientMessageHandler} uses a \texttt{Connection} and the constants defined in \texttt{Protocol} in order to send commands to the server and present the responses. Every action that the client can issue to the server is represented as a single method in the \texttt{ClientMessageHandler} class. For instance, it has methods such as \texttt{createNewsgroup}, \texttt{deleteNewsgroup}, \texttt{createArticle} and \texttt{getArticle} among others. This class also contains some private helper methods used to read and write strings and numbers received from and sent to the server.

The file named \texttt{Client} contains the \texttt{main}-method of the entire client and everything that is connected to the user interface and thus all interaction with the user. For example, this file contains methods such as \texttt{queryUserMenu}, \texttt{promptInt}, \texttt{promptString} and \texttt{interact}. These methods are fairly straightforward, the prompt methods are simply helper functions used to request input from the user. \texttt{queryUserMenu} displays to the user a menu of all available actions and asks for a selection. \texttt{interact} connects all functions together. It contains a loop that dislays the menu and depending on the selection it uses the prompt functions to gather input and finally it calls the appropriate method in the \texttt{ClientMessageHandler}. The \texttt{main}-method simply initializes the \texttt{ClientMessagehandler} and calls the \texttt{interact} function.

\section{Conclusions}

\textbf{Enligt projektbeskrivning:
"Requirements that you fulfill, problems that you haven’t succeeded in solving,
etc. If you have found that the system ought to have more features you should elaborate on
this. Any suggestions for improvements to the project are also welcome"}
\end{document}
